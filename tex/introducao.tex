\chapter{Introdução}

O automóvel, desdo seu surgimento, mudou drasticamente o modus operandi de se viver. Impactando diretamente na mobilidade, aumentando a capacidade de locomoção de seres vivos e objetos, aumentando o alcance de bens e serviços, os automóveis contribuíram e ainda contribuem significativamente para o desenvolvimento da nossa sociedade. Porém junto ao poder de alcançar o que está mais longe de forma mais rápida, o risco de acidentes em alta velocidade no contexto automotivo também está presente \cite{Malik2008}.
A busca pela segurança e conforto no contexto automotivo motivou, nas últimas décadas, o desenvolvimento de uma variedade de sistemas de assistência ao motorista. Como exemplo desses sistemas: Adaptive  Cruise  Control, Blind-Spot  Monitoring, Forward Collision Warning, Automatic Emergency Braking, Lane Departure Warning \cite{tsai2018}. Além dos sistemas de assistência a motoristas, sistemas de carros autônomos também podem se beneficiar da visão computacional \cite{Gehrig1999}. 
Para evitar acidentes e colisões, dentro do universo da visão computacional, existem diferentes técnicas para detecção de obstáculos. Recentemente, com a modernização dos processadores e câmeras de vídeo, os sistemas de visão computacional começam a se tornar uma opção promissora \cite{tsai2018}. Algumas técnicas utilizadas se baseiam no conceito de reconhecimento de imagem (para o reconhecimento da posição de um possível obstáculo ou caminho a ser tomado). Para o reconhecimento de imagem atualmente estão sendo bastante utilizadas técnicas de \textit{machine learning}\cite{Shanmugasundaram2018} e \textit{deep learning} \cite{Wang2018}. Porém essas técnicas são ineficientes para lidar com tipos de obstáculos que não estavam inclusos na base de treinamento durante o aprendizado do sistema de inteligência artificial \cite{tsai2018}. 

No contexto da visão computacional, uma alternativa aos métodos de inteligência artificial para detecção de obstáculos são os algoritmos baseados no conceito de visão estéreo \cite{bertozzi1996}. Similar à forma como seres humanos percebem a profundidade do ambiente tridimensional \cite{Marr1977}, os sistemas de visão estéreo utilizam duas câmeras em paralelo para extrair a informação de  profundidade da cena. A disparidade da posição dos objetos nas imagens capturadas por essas câmeras, devido à distância espacial e alinhamento entre as elas, carrega a informação da profundidade da cena \cite{lu2013}. Os sistemas de visão estéreo já se mostraram muito eficientes no contexto automotivo \cite{bertozzi1996}. Porém para que as estimativas de profundidade sejam precisas, a posição espacial e o alinhamento entre câmeras também devem ser conhecidos de forma precisa. Os sinais de vibrações que afetariam o posicionamento das câmeras, causados por diferentes tipos de solos, afetariam o desempenho do sistema. Pois o alinhamento entre câmeras que compõem o sistema de visão estéreo é crucial para o funcionamento do mesmo \cite{Setyawan2018}. Existem formas eliminar vibrações ruidosas de forma mecânica, porém tais mecanismos (exemplo: amortecedores) acabam sendo ineficientes para lidar com algumas faixas de frequência. Principalmente aquelas mais baixas (1 – 25Hz). Porém estabilizadores digitais apresentaram melhor solução para lidar com essas vibrações (Bombini, 2006).
Além do setor automotivo e terrestre, a visão computacional também vem sendo estudada em diferentes ramos de aplicação da eletrônica embarcada. Como exemplo dessas aplicações temos submarinos autônomos \cite{naglak2018} e \textit{Drones} \cite{kyristsis2016}. E nesses outros contextos os problemas relacionados a ruídos mecânicos também sempre se mostram presentes \cite{Hitaka2001}.
Tendo em vista a variedade de possibilidades e problemas que podem afetar um sistema de visão computacional automotivo (ou de outro setor); sendo esse sistema de visão estereoscópica, inteligência artificial ou outro; o desenvolvimento de uma plataforma de prototipagem se mostra pertinente para o entendimento dessa tecnologia. Logo esse trabalho tem como objetivo o desenvolvimento de uma plataforma de prototipagem miniaturizada e de baixo custo que permite a avaliação de sistemas de visão computacional. 


\section{Justificativa}

Tendo em vista a variedade de problemas que podem afetar um sistema de visão computacional, sendo no contexto automotivo ou outro, o desenvolvimento de uma plataforma de prototipagem se mostra pertinente para estudo da visão computacional. Dessa forma será possível investigar diferentes técnicas de visão computacional em um dispositivo capaz de se mover. Será possível investigar o quão resiliente um sistema de visão computacional é aos ruídos mecânicos e elétricos gerados pelo deslocamento do mesmo em diferentes tipos de solo. Também será possível avaliar diferentes unidades de processamento no quesito consumo de energia e tempo de processamento. Mesmo que o sistema não satisfaça as restrições de tempo real demandadas pela industria, esse será capaz de evidenciar os problemas que podem ocorrer em não atender os requisitos de tempo real.

\pagebreak

\section{Definição do Problema}

O tempo de resposta do sistema de visão computacional está relacionado com o tempo que o sistema leva para processar cada quadro. Logo se o sistema não for rápido o suficiente, não será capaz de responder corretamente aos eventos de maior velocidade. 
O sistema tem sua velocidade de resposta limitada pelo tempo de processamento e taxa de amostragem. 
O tempo de processamento é o tempo que leva para processar o sinal proveniente do sensoriamento do ambiente, de forma a identificar qual reação tomar.
Se o sistema demorar muito tempo para processar a informação, responderá o evento de forma proporcionalmente atrasada.
A taxa de amostragem é a quantidade de vezes por segundo que o sistema faz o sensoriamento do ambiente \cite{lathi2014}. 
O sistema pode não perceber os eventos que são rápidos os suficiente para acontecer entre os instantes de tempo onde acontece o sensoriamento. 
Logo conhecendo a tempo de processamento é possível definir a maior taxa de amostragem possível afim de obter o menor tempo de reação possível.
Além dos problemas no tempo de reação, outros problemas relacionados a ruídos no sinal amostrado podem deteriorar o desempenho do sistema. 
Baseado nesses problemas levantados, ao longo desse trabalho serão investigadas as seguintes perguntas.

\begin{alineas}
    \item Qual o tempo de resposta do sistema de visão computacional ste?
    \item É possível o desenvolvimento da plataforma de prototipagem de sistemas de visão computacional de baixo custo?
\end{alineas}

\section{Objetivo Geral}.

Desenvolver um ambiente de prototipagem de baixo custo para desenvolver e investigar sistemas móveis de visão computacional, sendo esse capaz de rodar sistemas que se deslocam autonomamente. 

\section{Objetivos Específicos}

\begin{alineas}
    \item Implementar carro remotamente controlável com funcionalidade de aquisição de imagem de forma simples ou estereoscópica.
    \item Implementar algoritmos para extrair o mapa de profundidade via sensoriamento estereoscópico.
    \item Avaliar o tempo de resposta do sistema com sensoriamento estereoscópico.
\end{alineas}

