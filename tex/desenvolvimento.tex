\chapter{DESENVOLVIMENTO}

\section{EXPOSIÇÃO DO TEMA OU MATÉRIA}

É a parte principal e mais extensa do trabalho. Deve apresentar a fundamentação teórica, a metodologia, os resultados e a discussão. Divide-se em seções e subseções conforme a NBR 6024~\cite{abnt14724}. Quanto a sua estrutura, segue as recomendações da norma para preparação de trabalhos acadêmicos, a NBR 14724 de 2011~\cite{abnt14724}. Quanto à Formatação, segue o modelo adotado pelo IFSC, o formato A4.


\subsection{Formatação do texto}


\subsubsection{As ilustrações}

Independente do tipo de ilustração (quadro, desenho, figura, fotografia, mapa, entre outros) sua identificação aparece na parte superior, precedida da palavra designativa. 

A indicação da fonte consultada deve aparecer na parte inferior, elemento obrigatório mesmo que seja produção do próprio autor. A ilustração deve ser citada no texto e inserida o mais próximo possível do texto a que se refere~\cite{abnt14724}. 

A Figura \ref{fig:a} mostra o logo da BU
\begin{figure}[!htb]
   \centering
   \caption{Logo da BU.}\label{fig:a}
   \includegraphics[width=0.3\textwidth]{./img/brasaoBU.jpg}
\end{figure}

A Tabela~\ref{tab:a} mostra mais informações do template BU.

\begin{table}[!htb]
\begin{center}
 \caption{Formatação do texto.}\label{tab:a}
  \begin{tabular}{ p{3cm} | p{6cm} }
    \hline
Cor & Branco\\ \hline
Formato do papel & A5\\ \hline
Gramatura & 75\\ \hline
Impressão & Frente e verso\\ \hline
Margens & Espelhadas: superior 2, Inferior: 1,5, Externa 1,5 e Externa: 2.\\ \hline
Cabeçalho & 0,7\\ \hline
Rodapé & 0,7\\ \hline
Paginação & Externa\\ \hline
Alinhamento vertical & Superior\\ \hline
Alinhamento do texto & Justificado\\ \hline
Fonte sugerida & Times New Roman \\ \hline
Tamanho da fonte & 10,5 para o texto incluindo os títulos das seções e subseções. As citações com mais de três linhas as legendas das ilustrações e tabelas, fonte 9,5.\\ \hline
Espaçamento entre linhas & Um (1) simples\\ \hline
Espaçamento entre parágrafos & Anterior 0,0; Posterior 0,0\\ \hline
Numeração da seção & As seções  primárias devem  começar  sempre em páginas ímpares. Deixar um espaço (simples) entre o título da seção e o texto e  entre o texto e o título da subseção. \\  \hline
  \end{tabular}
\end{center}
Fonte: Universidade Federal de Santa Catarina (2011)
\end{table}



\subsubsection{Equações e fórmulas}

As equações e fórmulas devem ser destacadas no texto para facilitar a leitura.  Para numerá-las, deve-se usar algarismos arábicos entre parênteses e alinhados à direita. Pode-se usar uma entrelinha maior do que a usada no texto~\cite{abnt14724}.

Exemplo: A equação \ref{eq:a}
\begin{equation}
 x^2 + y^2 = z^2
 \label{eq:a}
\end{equation}
 e a equação  \ref{eq:b}
\begin{equation}
 x^2 + y^2 = n
\label{eq:b}
\end{equation}


\subsubsection{Exemplo de citações no \LaTeX}

Segundo \citeonline{alves_2001} ...

...no final da frase \cite{abnt14724,BU_formatoA5}


\nocite{alves_2001,abnt10520,abnt6024,abnt14724}
