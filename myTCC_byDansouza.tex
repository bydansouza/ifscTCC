\documentclass{ifscTCC} % Definicao do documentclass ifscTCC	

% ----------------------------------------------------------
% Informações de dados para CAPA e FOLHA DE ROSTO
% ----------------------------------------------------------
% Identificadores do trabalho - Usados para preencher os elementos pré-textuais
\instituicao[o]{Instituto Federal de Educação, Ciência e Tecnologia de Santa Catarina} % Opcional
\departamento[o]{Departamento Acadêmico de Eletrônica}
\curso[o]{Curso Superior de Engenharia Eletrônica}
\documento[o]{TCC} % [o] para dissertação [a] para tese
\preambulo{Trabalho de conclusão de curso submetido ao Instituto Federal de Educação, Ciência e Tecnologia de Santa Catarina como parte dos requisitos para obtenção do título de engenheiro eletrônico}
\titulo{Estudo e caracterização de pontas de prova para obtenção de medidas de campo magnético próximo em análises de compatibilidade eletromagnetica}
%\subtitulo{Subtítulo (se houver)} % Opcional
\autor{Daniel Henrique Camargo de Souza}
\grau{Bacharel}
\local{Florianópolis} % Opcional (Florianópolis é o padrão)
\data{2019}
%\orientador[Orientador:\\]{Prof. Dr. Luis Carlos Martinhago Schlichting}
%\coorientador[Coorientador\\Universidade ...]{Prof. Dr.}
%\coordenador[Coordenador\\Universidade ...]{Prof. Dr. }

%---------------------------------------------------------------------
% Início do documento
%---------------------------------------------------------------------
\begin{document}
% Retira espaço extra obsoleto entre as frases.
\frenchspacing 


% ----------------------------------------------------------
% CAPA SEGUINDO NORMA TCC IFSC
% ----------------------------------------------------------
%\imprimircapa  %Comando Padrão do Abntex2 (Formato Diferente do IFSC)
\begin{capa}%
    \begin{SingleSpacing}
        \center\ABNTEXchapterfont\bfseries\ INSTITUTO FEDERAL DE EDUCAÇÃO, CIÊNCIA E TECNOLOGIA DE\\SANTA CATARINA - CÂMPUS FLORIANÓPOLIS\\DEPARTAMENTO ACADÊMICO DE ELETRÔNICA\\CURSO SUPERIOR DE ENGENHARIA ELETRÔNICA
        
        \vspace*{3.0cm}     % três espaços simples - conforme norma para TCC do IFSC
        
        \ABNTEXchapterfont\bfseries\MakeUppercase\imprimirautor

        \begin{vplace}[0.5]
            \begin{center}
                \ABNTEXchapterfont\SingleSpacing\bfseries\Large\MakeUppercase\imprimirtitulo
            \end{center}
        \end{vplace}
        
        \begin{center}
            \ABNTEXchapterfont\bfseries\ FLORIANÓPOLIS, 2019
        \end{center}
    \end{SingleSpacing}
\end{capa}

% ----------------------------------------------------------
% FOLHA DE ROSTO SEGUINDO NORMA TCC IFSC
% ----------------------------------------------------------
% Comando Padrão do Abntex2 % (o * indica que haverá a ficha bibliográfica) (Formato Diferente do IFSC)
% \imprimirfolhaderosto*
% \imprimirfolhaderosto
\begin{capa}%
    \begin{SingleSpacing}
        \center
        \ABNTEXchapterfont\bfseries\ INSTITUTO FEDERAL DE EDUCAÇÃO, CIÊNCIA E TECNOLOGIA DE\\SANTA CATARINA - CÂMPUS FLORIANÓPOLIS\\DEPARTAMENTO ACADÊMICO DE ELETRÔNICA\\CURSO SUPERIOR DE ENGENHARIA ELETRÔNICA
        
        \vspace*{3.0cm}     % três espaços simples - conforme norma para TCC do IFSC
        
        \ABNTEXchapterfont\bfseries\MakeUppercase\imprimirautor
        
        %\begin{vplace}[0.5]
        \vspace*{\fill} 
            \begin{center}
                \ABNTEXchapterfont\SingleSpacing\bfseries\Large\MakeUppercase\imprimirtitulo
            \end{center}
        \vspace*{\fill} 
        %\end{vplace}
        
        \hspace{.45\textwidth}
        \begin{minipage}{.5\textwidth}
            \begin{SingleSpacing}
                \normalfont\imprimirpreambulo
                \vspace*{1.0cm}
                
                Orientador:\\Prof. Dr. Luis Carlos Martinhago Schlichting
            \end{SingleSpacing}
        \end{minipage}%
        \vspace*{\fill} 
        
        \begin{center}
            \ABNTEXchapterfont\bfseries\ FLORIANÓPOLIS, 2019
        \end{center}
    \end{SingleSpacing}
\end{capa}

% ----------------------------------------------------------
% FICHA DE IDENTIFICAÇÃO (CATALOGRÁFICA) APROVAÇÃO SEGUINDO NORMA TCC IFSC
% ----------------------------------------------------------
\includepdf{./pdf/fichacatalografica_final.pdf}     %Para Inclusão de modelo fornecido pela Biblioteca
\cleardoublepage

%\begin{fichacatalografica}
%    \sffamily
%    \vspace*{15cm}    % Posição vertical
%    \hrule    % Linha horizontal
%    
%    \begin{center}
%        % Minipage Centralizado
%        \begin{minipage}[c]{12.5cm} % Largura
%            \imprimirautor
%            \hspace{0.5cm} \imprimirtitulo / \imprimirautor. --
%            \imprimirlocal, \imprimirdata-
%            \hspace{0.5cm} \pageref{LastPage} p. : il.(alguma color.); 30 cm.%\\
%            \hspace{0.5cm} \imprimirorientadorRotulo \imprimirorientador\\
%            \hspace{0.5cm}
%            \parbox[t]{\textwidth}{\imprimirtipotrabalho~--~\imprimirinstituicao,
%            \imprimirdata.}\\
%            \hspace{0.5cm}
%            1. Palavra-chave1.
%            2. Palavra-chave2.
%            I. Orientador.
%            II. Universidade xxx.
%            III. Faculdade de xxx.
%            IV. Título\\
%            \hspace{8.75cm} CDU 02:141:005.7\\
%        \end{minipage}
%    \end{center}
%    \hrule
%\end{fichacatalografica}

% ----------------------------------------------------------
% ERRATA SEGUINDO NORMA TCC IFSC (OPCIONAL)
% ----------------------------------------------------------
%\begin{errata}
%    FERRIGNO, C. R. A. \textbf{Tratamento de neoplasias ósseas apendiculares com reimplantação de enxerto ósseo autólogo autoclavado associado ao plasma rico em plaquetas}: estudo crítico na cirurgia de preservação de membro em cães. 2011. 128 f. Tese (Livre-Docência) - Faculdade de Medicina
%    Veterinária e Zootecnia, Universidade de São Paulo, São Paulo, 2011.
%    \begin{table}[htb]
%        \center
%        \footnotesize
%        \begin{tabular}{|p{1.4cm}|p{1cm}|p{3cm}|p{3cm}|}
%            \hline
%            \textbf{Folha} & \textbf{Linha} & \textbf{Onde se lê} &
%            \textbf{Leia-se}\\
%            \hline
%            1 & 10 & auto-conclavo & autoconclavo\\
%            \hline
%        \end{tabular}
%    \end{table}
%\end{errata}
%
% ----------------------------------------------------------
% FOLHA APROVAÇÃO SEGUINDO NORMA TCC IFSC
% ----------------------------------------------------------
%\includepdf{folhadeaprovacao_final.pdf}     %Após as assinaturas é incluido digitalizado

\begin{folhadeaprovacao}
    \begin{center}
        \begin{center}
            \ABNTEXchapterfont\SingleSpacing\bfseries\Large\MakeUppercase\imprimirtitulo
        \end{center}
            
        \vspace*{2.0cm}
            
        \ABNTEXchapterfont\normalsize\bfseries\MakeUppercase\imprimirautor
            
        \vspace*{2.0cm}
    \end{center}
    
    
    \noindent Este Trabalho foi julgado adequado para obtenção do Título de Engenheiro Eletrônico em XXX e aprovado na sua forma final pela banca examinadora do Curso de Engenharia Eletrônica do instituto Federal de Educação Ciência, e Tecnologia de Santa Catarina.
        
    \vspace*{1.0cm}
    \begin{center}    
        \imprimirlocal, 04 de Julho de 2019.
    \end{center}
    
    \noindent Banca Examinadora:
    
    \assinatura{Luis Carlos Martinhago Schlichting, Dr.}
    \assinatura{Convidado 1, Dr.}
    \assinatura{Convidado 2, Dr.}
    \assinatura{Convidado 3, Dr.}
    \assinatura{Convidado 4, Dr.}

\end{folhadeaprovacao}
\cleardoublepage

% ----------------------------------------------------------
% DEDICATORIA SEGUINDO NORMA TCC IFSC
% ----------------------------------------------------------
\begin{dedicatoria}

    \vspace*{\fill}
        Para Voçê!
    \vspace*{\fill}
    
\end{dedicatoria}


% ----------------------------------------------------------
% AGRADECIMENTOS SEGUINDO NORMA TCC IFSC
% ----------------------------------------------------------
\begin{agradecimentos}
    \begin{itemize}
        \item ao \LaTeX que coloca o Word no chinelo 
    \end{itemize}

\end{agradecimentos}


% ----------------------------------------------------------
% EPIGRAFE SEGUINDO NORMA TCC IFSC
% ----------------------------------------------------------
%\begin{epigrafe}
%	
%    \noindent\textit{‘‘Não vos amoldeis às estruturas deste mundo, \\
%    mas transformai-vos pela renovação da mente, \\
%    a fim de distinguir qual éa vontade de Deus: \\
%    o que é bom, o que Lhe éagradável, o que éperfeito.\\
%    (Bíblia Sagrada, Romanos 12, 2)}
%    
%\end{epigrafe}

% ----------------------------------------------------------
% RESUMO SEGUINDO NORMA TCC IFSC
% ----------------------------------------------------------
% resumo na língua vernácula (obrigatório)
\setlength{\absparsep}{18pt} % ajusta o espaçamento dos parágrafos do resumo

\begin{resumo}



\noindent
\textbf{Palavras-chaves}: Compatibilidade. 

\end{resumo}

% ----------------------------------------------------------
% ABSTRACT SEGUINDO NORMA TCC IFSC
% ----------------------------------------------------------
\begin{resumo}[ABSTRACT]
\begin{otherlanguage*}{english}



\noindent
\textbf{Key-Words}: Compatibility. 
\end{otherlanguage*}
\end{resumo}

% ----------------------------------------------------------
% LISTA DE ILUSTRAÇÕES
% ----------------------------------------------------------
\pdfbookmark[0]{\listfigurename}{lof}
\listoffigures*
\cleardoublepage

% ----------------------------------------------------------
% LISTA DE TABELAS
% ----------------------------------------------------------
\pdfbookmark[0]{\listtablename}{lot}
\listoftables*
\cleardoublepage

% ----------------------------------------------------------
% LISTA DE ABREVIATURAS E SIGLAS
% ----------------------------------------------------------
\begin{siglas}
   \item[VDC] \textit{Voltage Direct Current} - Tensão Contínua
   \item[VAC] \textit{Voltage Alternating Current} - Tensão Alternada
   \item[CI] Circuito Integrado
   \item[CC] Corrente Contínua
   \item[CA] Corrente Alternada
   \item[IFSC] Instituto Federal de Educação Ciência e Tecnologia de Santa Catarina
   \item[PCI] Placa de Circuito Impresso
   \item[ABNT] Associação Brasileira de Normas Técnicas
   \item[abnTex] Normas para \LaTeX
   \end{siglas}

% ----------------------------------------------------------
% LISTA DE SIMBOLOS
% ----------------------------------------------------------
\begin{simbolos}
   \item[$cm$] Centímetros - Unidade de comprimento
   \item[$cm^{2}$] Centímetros Quadrados - Unidade de área
   \item[$VA$] Volt-Ampere - Unidade de potência elétrica
   \item[$W$] Watt's - Unidade de potência elétrica
   \item[$V$] Volts - Unidade de potencial elétrico
 	\item[$A$] Ampere - Unidade de Corrente Elétrica
 	\item[$\Omega$] Ohms - Unidade de resistência elétrica
 	\item[$H$] Henry - Unidade de indutância elétrica
 	\item[$\,^{\circ}\mathrm{C}$] Grau Celcius - Unidade de temperatura
\end{simbolos}

% ----------------------------------------------------------
% SUMÁRIO
% ----------------------------------------------------------
\pdfbookmark[0]{\contentsname}{toc}
\tableofcontents
\cleardoublepage

% ----------------------------------------------------------
% ELEMENTOS TEXTUAIS
% ----------------------------------------------------------
\textual

% Cores para os códigos do MatLab
\definecolor{mygreen}{RGB}{28,172,0}    % color values Red, Green, Blue
\definecolor{mylilas}{RGB}{170,55,241}  % o lilas para códigos do MatLab

\lstset{language=Matlab,            %Que tipo de linguaguem os códigos serão
    %basicstyle=\color{red}, 		%normal fontsize
    %basicstyle=\ttfamily\footnotesize 	%fontsize small
    basicstyle=\ttfamily\scriptsize,		%fontsize verysmall
    %breaklines=false,%
    morekeywords={matlab2tikz},
    keywordstyle=\color{blue},%
    morekeywords=[2]{1}, keywordstyle=[2]{\color{black}},
    identifierstyle=\color{black},%
    stringstyle=\color{mylilas},
    commentstyle=\color{mygreen},%
    showstringspaces=false,%without this there will be a symbol in the places where there is a space
    numbers=left,%
    numberstyle={\tiny \color{black}},% size of the numbers
    numbersep=5pt, % this defines how far the numbers are from the text
    emph=[1]{for,end,break},emphstyle=[1]\color{red}, %some words to emphasise
    %emph=[2]{word1,word2}, emphstyle=[2]{style},    
}

% Elimina o Cabeçalho
\pagestyle{parpage}

%\part{Introdução}
 \chapter{Introdução}
Suspendisse potenti. Aenean sagittis ante in aliquet fringilla. Integer pellentesque consequat nisl sed eleifend. Nunc vulputate eleifend ligula, sit amet fermentum mauris elementum id. Sed felis magna, pharetra sed lorem id, dapibus facilisis lorem. Donec eleifend faucibus eros nec tincidunt. Integer blandit ligula vel metus molestie, eu congue justo tristique. Aliquam luctus lorem tristique odio feugiat consequat. Aenean eleifend risus eros, eu fermentum urna tempus at. Proin tempus leo eros, a vulputate dolor mattis ac. Suspendisse a turpis cursus, commodo libero in, auctor massa. Pellentesque urna metus, mattis id venenatis et, consequat eget mi. Vestibulum maximus eu massa sit amet pulvinar. Mauris a pretium sem. Duis sed purus a magna blandit pharetra eget ullamcorper nisl. Morbi blandit nunc at quam maximus venenatis.

Suspendisse potenti. Aenean sagittis ante in aliquet fringilla. Integer pellentesque consequat nisl sed eleifend. Nunc vulputate eleifend ligula, sit amet fermentum mauris elementum id. Sed felis magna, pharetra sed lorem id, dapibus facilisis lorem. Donec eleifend faucibus eros nec tincidunt. Integer blandit ligula vel metus molestie, eu congue justo tristique. Aliquam luctus lorem tristique odio feugiat consequat. Aenean eleifend risus eros, eu fermentum urna tempus at. Proin tempus leo eros, a vulputate dolor mattis ac. Suspendisse a turpis cursus, commodo libero in, auctor massa. Pellentesque urna metus, mattis id venenatis et, consequat eget mi. Vestibulum maximus eu massa sit amet pulvinar. Mauris a pretium sem. Duis sed purus a magna blandit pharetra eget ullamcorper nisl. Morbi blandit nunc at quam maximus venenatis.

\section{OBJETIVOS}
Suspendisse potenti. Aenean sagittis ante in aliquet fringilla. Integer pellentesque consequat nisl sed eleifend. Nunc vulputate eleifend ligula, sit amet fermentum mauris elementum id. Sed felis magna, pharetra sed lorem id, dapibus facilisis lorem. Donec eleifend faucibus eros nec tincidunt. Integer blandit ligula vel metus molestie, eu congue justo tristique. Aliquam luctus lorem tristique odio feugiat consequat. Aenean eleifend risus eros, eu fermentum urna tempus at. Proin tempus leo eros, a vulputate dolor mattis ac. Suspendisse a turpis cursus, commodo libero in, auctor massa. Pellentesque urna metus, mattis id venenatis et, consequat eget mi. Vestibulum maximus eu massa sit amet pulvinar. Mauris a pretium sem. Duis sed purus a magna blandit pharetra eget ullamcorper nisl. Morbi blandit nunc at quam maximus venenatis.

\subsection{Objetivo Geral}

Suspendisse potenti. Aenean sagittis ante in aliquet fringilla. Integer pellentesque consequat nisl sed eleifend. Nunc vulputate eleifend ligula, sit amet fermentum mauris elementum id. Sed felis magna, pharetra sed lorem id, dapibus facilisis lorem. Donec eleifend faucibus eros nec tincidunt. Integer blandit ligula vel metus molestie, eu congue justo tristique. Aliquam luctus lorem tristique odio feugiat consequat. Aenean eleifend risus eros, eu fermentum urna tempus at. Proin tempus leo eros, a vulputate dolor mattis ac. Suspendisse a turpis cursus, commodo libero in, auctor massa. Pellentesque urna metus, mattis id venenatis et, consequat eget mi. Vestibulum maximus eu massa sit amet pulvinar. Mauris a pretium sem. Duis sed purus a magna blandit pharetra eget ullamcorper nisl. Morbi blandit nunc at quam maximus venenatis.

\subsection{Objetivos Específicos}

Suspendisse potenti. Aenean sagittis ante in aliquet fringilla. Integer pellentesque consequat nisl sed eleifend. Nunc vulputate eleifend ligula, sit amet fermentum mauris elementum id. Sed felis magna, pharetra sed lorem id, dapibus facilisis lorem. Donec eleifend faucibus eros nec tincidunt. Integer blandit ligula vel metus molestie, eu congue justo tristique. Aliquam luctus lorem tristique odio feugiat consequat. Aenean eleifend risus eros, eu fermentum urna tempus at. Proin tempus leo eros, a vulputate dolor mattis ac. Suspendisse a turpis cursus, commodo libero in, auctor massa. Pellentesque urna metus, mattis id venenatis et, consequat eget mi. Vestibulum maximus eu massa sit amet pulvinar. Mauris a pretium sem. Duis sed purus a magna blandit pharetra eget ullamcorper nisl. Morbi blandit nunc at quam maximus venenatis.

 \chapter{Fundamentação Teórica}

\section{Linux Embarcado}

\subsection{\textit{Raspberry Pi 3 Model B+} - Versão Anatel}

O \textit{Raspberry Pi 3 Model B+} é um \textit{kit} de desenvolvimento que carrega como principal elemento o \textit{SOC} \textit{Broadcom BCM2837B0}.
O processador que esse \textit{SOC} carrega é um \textit{ARM Cortex-A53} com quatro núcleos \textit{64-bits} que rodam com um \textit{clock} de $1.4 GHz$ \cite{rasp3bplus2019}. 
Existe a versão do \textit{kit} de desenvolvimento homologada pela Anatel. A versão da placa homologada pela Anatel é azul como a Figura \ref{fig:rasp_anatel}. Essa versão homologada foi produto de uma parceria com a fundação do projeto Raspberry Pi com a revendedora de componentes eletrônicos catarinense Felipe Flop \cite{raspanatel2019}.

\begin{figure}[!htb]
    \centering
    \caption{\textit{Raspberry Pi 3 Model B+} - Versão Anatel.}
    \label{fig:rasp_anatel}
    \includegraphics[width=0.6\textwidth]{./img/fundamentacao/rasp_anatel.jpg}
\end{figure}
Fonte: Filipe Flop

O \textit{SOC} \textit{Cypress CYW43455} é o componente encarregado pelo sistema de \textit{WIFI} e \textit{Bluetooth} 4.2. 
Esse componente fornece uma banda de $46.7Mb/s$ na transmissão de datos e $46.3Mb/s$ na recepção de dados com a frequência de portadora $2.4GHz$ e $102Mb/s$ transmissão e recepção de dados com frequência de portadora igual $5GHz$. 
O barramento do \textit{USB} 2.0 e a rede \textit{10/100 Ethernet} é responsabilidade do componente \textit{LAN7515} da \textit{Microchip}. As velocidades levantadas pelo fabricante do \textit{Raspberry Pi} para esse barramento de $315Mb/s$  com frequência de portadora igual $5GHz$ \cite{rasp3bplus2019}.
 
 \pagebreak
 
\subsection{Ubuntu Mate 18.04 e Raspberry Pi}
O \textit{Ubuntu MATE 18.04 (Bionic Beaver) Beta 1}, avaliado para \textit{Raspberry Pi Model B 2}, \textit{3} and \textit{3+}, é disponibilizado na forma já compilada para as arquiteturas \textit{armhf (ARMv7 32-bit)} e \textit{arm64 (ARMv8 64-bit)}.
Esse sistema provê, além dos \textit{drivers} para \textit{Ethernet}, \textit{WIFI} e \textit{Bluetooth}, acesso ao \textit{GPIO} via \textit{GPIO Zero}, \textit{pigpio} e \textit{WiringPi} \cite{wimpress2019}.   

\subsection{\textit{ROS Melodic Morenia} e \textit{Ubuntu Mate 18.04}}

O \textit{ROS Melodic Morenia} é direcionado principalmente para a versão \textit{Ubuntu 18.04 (Bionic)}, embora outros sistemas \textit{Linux} e \textit{Mac OS X}, \textit{Android} e \textit{Windows} sejam suportados em vários graus \cite{rosmelodic2019}.
Logo o \textit{Ubuntu Mate 18.04} se mostra uma boa alternativa para o uso do sistema \textit{ROS} junto ao\textit{raspberry Pi 3 Model B+}.

\subsection{Configuração do modo \textit{Access Point}} - \textit{Netplan}
O \textit{Netplan} é o \textit{software} nativo do \textit{Ubuntu Mate 18.04}.
Com um arquivo de configuração de extensão \textit{.yaml}, por meio desse \textit{software}, é possível configurar o  

\subsection{\textit{Python} Embarcado}

\subsection{\textit{OpenSSH-server}}

\pagebreak

\section{\textit{OpenCV}}
A biblioteca de \textit{software} \textit{OpenCV} (\textit{Open Source Vision Library}) é uma biblioteca de código aberto desenvolvida pela comunidade afim de otimizar o processo de produção de \textit{softwares} de processamento de imagem. Essa biblioteca conta com mais de 2500 algorítimos otimizados de visão computacional e \textit{machine learning}.
Essa biblioteca é amplamente utilizada em empresas como \textit{Google}, \textit{Yahoo}, \textit{Microsoft}, \textit{Intel}, \textit{IBM}, \textit{Sony}, \textit{Honda}, \textit{Toyota} \cite{aboutopencv2019}. Sendo um produto de licença do tipo \textit{BSD-licensed} é uma alternativa para desenvolvimento de produtos comerciais \cite{opensource2019}.
    
Essa biblioteca é escrita nativamente em \textit{C++}, porém fornece interfaces para as linguagens de programação \textit{C++}, \textit{Python}, \textit{Java} e \textit{Matlab} e da suporte para os sistemas operacionais \textit{Windows}, \textit{Linux}, \textit{Mac OS} e \textit{Android}.  

\subsection{Compilação do \textit{OpenCV}}

\begin{itemize}
    \item \textbf{Ferramentas para compilação:} \textit{build-essential}, \textit{cmake}, \textit{unzip} e \textit{pkg-config}.
    \item \textbf{Bibliotecas de vídeo e imagem:} \textit{libjpeg-dev}, \textit{libpng-dev}, \textit{libtiff-dev}, \textit{libavcodec-dev}, \textit{libavformat-dev},  \textit{libswscale-dev}, \textit{libv4l-dev}, \textit{libxvidcore-dev} e \textit{libx264-dev}.
    \item \textbf{GUI-\textit{Backend}:} \textit{libgtk-3-dev} e \textit{libcanberra-gtk*}.
    \item \textbf{Otimização Numérica:} \textit{libatlas-base-dev} e \textit{gfortran}.
    \item \textbf{Arquivos de cabeçalho para criação de pacotes python:} \textit{python3-dev}.
\end{itemize}

\cite{rosebrock2018}

% Ferramentas para compilação:\
% $$ \$ sudo apt-get install build-essential cmake unzip pkg-config $$
% Next, let’s install a selection of image and video libraries — these are critical to being able to work with image and video files:
% $$ \$ sudo apt-get install libjpeg-dev libpng-dev libtiff-dev $$
% $$ \$ sudo apt-get install libavcodec-dev libavformat-dev libswscale-dev libv4l-dev $$
% $$ \$ sudo apt-get install libxvidcore-dev libx264-dev $$
% From there, let’s install GTK, our GUI backend:
% $$ \$ sudo apt-get install libgtk-3-dev $$

\pagebreak

\subsection{\textit{OpenCV} - \textit{Python}}
Todos os algorítimos da biblioteca \textit{OpenCV} são implementados com a linguagem de programação \textit{C++}. 
Porém a programação dessa biblioteca com o uso de outras linguagens de programação se faz possível por meio de um mecanismo que a documentação do \textit{OpenCV} intitula como \textit{bindings generator}. 
Esse gerador é responsável por gerar um mecanismo que possibilita ao interpretador \textit{Python} fazer chamadas de funções já compiladas que foram escritas com a linguagem \textit{C++}.
Essas pontes (\textit{bindings}) poderiam ser programadas manualmente, uma a uma, por um programador. 
Porém pelo fato da biblioteca conter mais de 2500 funções \cite{aboutopencv2019}, a programação manual dessas pontes se torna uma tarefa demasiadamente exaustiva.
A forma que os desenvolvedores do \textit{OpenCV} adotaram para viabilizar a criação de todas as pontes, uma para cada função da biblioteca, foi a automatização do processo. 
Um \textit{script} escrito com a linguagem de programação \textit{python} faz a leitura de todos os arquivos de cabeçalhos (\textit{headers}) da biblioteca \textit{OpenCV} escrita em \textit{C++} e gera uma ponte (\textit{binding}) para cada função da biblioteca\cite{opencvpython2019}.
Dessa forma é possível utilizar da eficiência de uma função já compilada junto a praticidade de uma linguagem interpretada.
  
\pagebreak

\section{\textit{ROS}}

O sistema \textit{ROS}, \textit{Robot Operating System} é um  \textit{frameworks} para desenvolvimento de robôs. O \textit{ROS} fornece a funcionalidade de um sistema operacional que é capaz de operar múltiplos computadores heterogêneo \cite{aboutros2019}. Isso é pertinente para atender os objetivos específicos desse trabalho referentes a comunicação \textit{TCP/IP}.
O sistema \textit{ROS} é inicialmente composto por um nó central acionado pelo comando \textit{roscore}.
Após a inicialização desse nó central, mais nós podem ser iniciados formando uma rede com protocolo \textit{TCP/IP}.
Esses nós podem rodar dentro de um mesmo computador ou em diferentes computadores.
O nó central, iniciado pelo comando \textit{roscore}, é responsável por gerir a rede, e quando ele é desativado, todos os nós são desativados. 

A Figura \ref{fig:roscore} mostra a saída do programa \textit{roscore}. Ou seja, a mensagem que o programa escreve após sua execução. 
No caso do sistema utilizado nesse trabalho, a sua distribuição é chamada de \textit{Melodic Morenia 1.14.3}. 
Essa é a décima segunda versão. 
Sua data de lançamento foi 23 de maio de 2018 \cite{rosmelodic2019}.
De ano em ano uma distribuição é lançada e essa recebe suporte técnico de 5 anos \cite{rosdist2019}.
Na Figura \ref{fig:roscore} também é possível ver também o endereço do nó principal: \textbf{ROS\_MASTER\_URI=http://lucas-K46CA:11311/} e o \textit{pid} do processo.

\begin{figure}[!htb]
    \centering
    \caption{Comando \textit{roscore} - Inicialização do nó principal da rede ROS.}
    \label{fig:roscore}
    \includegraphics[width=\textwidth]{./img/fundamentacao/roscore.png}
\end{figure}
Fonte: Autor

\pagebreak

Sendo a ideia do \textit{ROS} a operação de vários nós em uma rede operando com protocolo \textit{TCP/IP}, cada nó acaba recebendo uma identificação. 
Considerando uma exemplificação adaptada de \citeonline{fernandes2019}, onde uma rede com três nós está operando, o comando \textit{rosnode list} tem como saída a listagem do nome dos três nós em operação. 
A Figura \ref{fig:rosnode_list} mostra saída do comando \textit{rosnode list}. 
A listagem dos três nós, sendo o nó com nome \textbf{\textit{/talker}} programado para enviar mensagens, o nó com nome \textbf{\textit{/listener}} programado para escutar mensagens e o nó com nome \textbf{\textit{/rosout}} é o \textit{console log reporting}. 
Esse é responsável por publicar e receber mensagem do sistema. 
O nó \textbf{\textit{/rosout}} sempre é iniciado com o comando \textit{roscore}.

\begin{figure}[!htb]
  \centering
  \caption{Comando \textit{rosnode list} - lista os nós ativos.}
  \label{fig:rosnode_list}
  \includegraphics[width=0.9\textwidth]{./img/fundamentacao/rosnode_list.png}
\end{figure}

Fonte: Autor

Os nós podem interagir entre si por meio de mensagens escritas em tópicos (\textit{topics}) ou por meio de \textit{services} que alguns nós venham a fornecer.
Os tópicos são uma abstração de locais na rede onde é possível publicar ou ler mensagens. 
A Figura \ref{fig:rostopic_list} mostra como a execução do comando \textit{rostopic list} mostra todos os tópicos ativos. 
No caso, o nó \textit{/talker} é programado para escrever no tópico \textit{/chatter} enquanto o nó \textit{listener} é programado para ler o tópico \textit{/chatter}. 
Dessa forma o nó \textit{/talker} pode enviar mensagens de maneira assíncrona para o nó \textit{/listener}.

\begin{figure}[!htb]
  \centering
  \caption{Comando \textit{rostopic list} - lista os tópicos ativos.}
  \label{fig:rostopic_list}
  \includegraphics[width=0.9\textwidth]{./img/fundamentacao/rostopic_list.png}
\end{figure}

Fonte: Autor

Por meio da abstração dos tópicos é possível definir a rota das mensagens pela rede. Vale ressaltar que um tópico é uma abstração. 
As mensagens não são enviadas a um tópico com endereço específico na rede. 
As mensagens são enviadas diretamente de um nó a outro. 
Os tópicos servem para facilitar a descrição do fluxo de dados da rede.

\pagebreak

O comando \textbf{\textit{rosnode info /roscore}}, como indicado na Figura \ref{fig:rosnode_info_rosout}, tem como saída informações pertinentes sobre cada nó, no caso o nó \textit{/roscore}.
Em \textit{Publications} é indicado em quais tópicos esse nó escreve mensagens e em \textit{Subscriptions} é indicado em quais tópicos esse nó lê mensagens.
Dessa forma é possível ver que o nó \textit{/rosout} está inscrito no tópico \textit{/rosout} (que possui o mesmo nome) e publica mensagens no tópico \textit{/rosout\_agg}.
Também é possível visualizar o tipo da mensagem que o tópico utiliza pois esse é tipado (o dado possui formato específico).
No caso do tópico \textit{/rosout}, esse opera com mensagens do tipo \textit{rosgraph\_msgs/Log}.

Como mencionado anteriormente, além da escrita em tópicos, uma outra forma dos nós interagirem é por meio dos \textit{services}. 
A Figura \ref{fig:rosnode_info_rosout} mostra os \textit{services} disponíveis pelo nó \textit{/rosout}. 
O \textit{get\_logger} e \textit{set\_logger\_level} são \textit{services} pertencentes ao nó \textit{/rosout} que podem ser acionados a qualquer momento.
Esses dois \textit{services} em específico são utilizados para depuração.

\begin{figure}[!htb]
  \centering
  \caption{Comando \textit{rosnode info /rosout} - Informações sobre o nó \textit{/rosout}.}
  \label{fig:rosnode_info_rosout}
  \includegraphics[width=0.9\textwidth]{./img/fundamentacao/rosnode_info_rosout.png}
\end{figure}
Fonte: Autor


Também é disposta todas as conexões que esse nó estabelece, na parte indicada com a palavra \textit{Connections} na Figura \ref{fig:rosnode_info_rosout}. 
É possível ver que o nó central se conecta tanto com o nó \textit{/listener} quanto com o nó \textit{/talker}, pelo tópico \textit{/rosout}. 
Também é indicado a direção da mensagem (\textit{inbound} significa que o nó recebe a mensagem) e o protocolo de transporte da mensagem, que no caso é uma variação do TCP/IP, o TCPROS.

\pagebreak

O comando \textbf{\textit{rosnode info /talker}} por sua vez mostra as informações do nó \textit{/talker}.
Na Figura \ref{fig:rosnode_info_talker} é possível verificar que esse nó escreve nos tópicos \textit{/chatter} e \textit{/rosout}. 
Diferente do nó \textit{/rosout} que lê as mensagens do tópico \textit{/rosout}, o nó \textit{/talker} escreve mensagens no tópico.
Logo o campo \textit{direction} está indicando \textit{outbound}, diferente do caso do nó \textit{/rosout}, Figura \ref{fig:rosnode_info_rosout}, que estava indicando \textit{inbound}.
A Figura \ref{fig:talker_rosnode_info_connect} mostra o grafo montado com as informações apresentadas na Figura \ref{fig:rosnode_info_talker}.

\begin{figure}[!htb]
  \centering
  \caption{Comando \textit{rosnode info /talker} - Informações sobre o nó \textit{/talker}.}
  \label{fig:rosnode_info_talker}
  \includegraphics[width=0.8\textwidth]{./img/fundamentacao/rosnode_info_talker.png}
\end{figure}
Fonte: Autor

\begin{figure}[!htb]
  \centering
  \caption{Conexões do nó \textit{/talker}.}
  \label{fig:talker_rosnode_info_connect}
  \includegraphics[width=0.50\textwidth]{./img/fundamentacao/talker_rosnode_info_connect.png}
\end{figure}
Fonte: Autor

\pagebreak

Com o comando \textbf{\textit{rosnode info /listener}} é possível extrair as mesmas informações apresentadas na Figura \ref{fig:rosnode_info_talker}, só que agora referentes ao nó \textit{/listener}. 
Dessa vez o campo \textit{direction} do tópico \textit{/chatter} está indicando \textit{inbound}. Isso significa que, diferente do nó \textit{/talker}, o nó \textit{/listener} lê mensagens do tópico \textit{/chatter}.

\begin{figure}[!htb]
  \centering
  \caption{Comando \textit{rosnode info /listener} - Informações sobre o nó \textit{/listener}.}
  \label{fig:rosnode_info_listener}
  \includegraphics[width=0.8\textwidth]{./img/fundamentacao/rosnode_info_listener.png}
\end{figure}
Fonte: Autor

Com as informações apresentadas na Figura \ref{fig:rosnode_info_rosout}, Figura \ref{fig:rosnode_info_talker} e Figura \ref{fig:rosnode_info_listener} é possível montar o grafo completo da rede (Figura \ref{fig:rosnet_listener_talker}).
Dessa forma resumidamente é demostrada de forma básica, por meio da exemplificação anterior, a lógica de funcionamento do ROS.

\begin{figure}[!htb]
  \centering
  \caption{Rede Completa.}
  \label{fig:rosnet_listener_talker}
  \includegraphics[width=0.50\textwidth]{./img/fundamentacao/rosnet_listener_talker.png}
\end{figure}
Fonte: Autor

\pagebreak

\subsection{ROS - TCP/IP - TCPROS}

A camada de transporte é chamada de TCPROS.
Nessa camada é utilizado o protocolo TCP/IP para transporte de mensagens via \textit{socket} \cite{tcpros2019}.
O ROS atualmente suporta protocolos do tipo TCP-IP e UDP-IP para estabelecer a comunicação entre nós \cite{rostopic2019}. 
E as mensagens são roteadas até o destino correto pelas informações contidas no \textit{TCPROS Header Field}.

\subsection{ROS - ROSSERIAL}

O pacote ROSSERIAL possibilita o uso de protocolo serial entre nós de uma rede ROS.
Esse pacote cria uma ponte entre o protocolo TCPROS(TCP/IP) e o protocolo serial. 
Dessa forma é possível trocar mensagens por meio de \textit{sockets} de rede \cite{fernandes_rosserial2019}. 
Atualmente o pacote ROSSERIAL fornece os seguintes suportes\cite{rosserial2019}:
\begin{itemize}
    \item \textbf{rosserial\_arduino} - suporte para plataforma Arduino.
    \item \textbf{rosserial\_embeddedlinux} - suporte para sistemas com Linux Embarcado (ex: Roteadores).
    \item \textbf{rosserial\_windows} - suporte para aplicações \textit{Windows}.
    \item \textbf{rosserial\_mbed} - suporte para plataforma \textit{Mbed}.
    \item \textbf{rosserial\_tivac} - suporte para a plataforma \textit{TI's Launchpad}.
    \item \textbf{rosserial\_vex\_v5} - suporte para o módulo \textit{VEX V5 Robot Brain}.
    \item \textbf{rosserial\_vex\_cortex} - suporte para o módulo \textit{VEX Cortex}.
    \item \textbf{rosserial\_stm32} - suporte para microcontroladores da linha \textit{STM32}.
    \item \textbf{ros-teensy} - suporte para a plataforma \textit{Teensy}.
\end{itemize}

Para o caso do uso da plataforma Arduino, que fonece suporte para o microcontrolador \textit{ATMEGA328P}, é compilada uma biblioteca que pode ser utilizada \cite{fernandes_rosserial2019}. 
Com um nó operando em um \textit{ATMEGA328P} é possível publicar em 25 tópicos diferentes e monitorar outros 25 tópicos. 
Dessa forma consumindo 560 bytes de memória (considerando os 50 tópicos - 25 para publicação e 25 para receber mensagens) \cite{arduinoros2019}. 

\pagebreak

\subsection{ROS e OpenCV}



\pagebreak
% \subsection{ROS - Python}

% \pagebreak

% \section{ROS e OpenCV}

% \pagebreak

% \pagebreak

% \section{HOG}

\section{Visão Estéreo}

Quando duas câmeras alinhadas capturam a imagem de um mesmo objeto, a disparidade da posição ocupada pelo objeto em cada sensor está relacionada com a profundidade do mesmo.
A Figura \ref{fig:stereo_vision_main_example} ilustra geometricamente como a disparidade das posições ocupadas nos sensores por um mesmo ponto pertencente ao seguimento $Z_d$ revela a profundidade do mesmo \cite{Kyto2011}.
Logo existe uma relação derivada do alinhamento entre todos os pontos da profundidade $Z_d$ com ponto de foco $f_L$ e todos os pontos do seguimento $Z_L$ no sensor ótico.

\begin{itemize}
    \item \textbf{$Z_d$}: Profundidade a ser medida.
    \item \textbf{$Z_R$}: Projeção de $Z_{d}$ no sensor ótico \textbf{direito} - \textit{Right}.
    \item \textbf{$Z_L$}: Projeção de $Z_{d}$ no sensor ótico \textbf{esquerdo} - \textit{Left}.
    \item \textbf{$f_R$}: Ponto focal da câmera \textbf{direita} - \textit{Right}.
    \item \textbf{$f_L$}: Ponto focal da câmera \textbf{esquerda} - \textit{Left}.
    \item \textbf{$b$}: Distância entre os pontos focais - \textit{baseline}.
\end{itemize}

\begin{figure}[!htb]
  \centering
  \caption{Sistema Estereoscópico.}
  \label{fig:stereo_vision_main_example}
  \includegraphics[width=0.65\textwidth]{./img/fundamentacao/stereo_vision_main_example.png}
\end{figure}
Fonte: Autor

\pagebreak

\subsection{Oclusão no contexto da visão estéreo.}

\ref{fig:stereo_vision_occlusion}
\begin{figure}[!htb]
  \centering
  \caption{Zona de oclusão.}
  \label{fig:stereo_vision_occlusion}
  \includegraphics[width=0.7\textwidth]{./img/fundamentacao/stereo_vision_occlusion.png}
\end{figure}
Fonte: Autor

\pagebreak

\subsection{Resolução de profundidade em função da distância focal.}

\ref{fig:stereo_vision_focal_variation}

$$dZ_d = \frac{Z^2}{f_{L, R}b}dp_x$$
$$dp_x = \frac{1}{2 tan(FOV)} S_w \Delta T$$

\begin{figure}[!htb]
  \centering
  \caption{Variação da distância focal.}
  \label{fig:stereo_vision_focal_variation}
  \includegraphics[width=0.80\textwidth]{./img/fundamentacao/stereo_vision_focal_variation.png}
\end{figure}

Fonte: Autor

\pagebreak

\subsection{Resolução de profundidade em função da variação da distância entre câmeras - \textit{baseline}.}

Figura \ref{fig:stereo_vision_bx_variation}

\begin{figure}[!htb]
  \centering
  \caption{Variação da \textit{baseline} ao longo do eixo x.}
  \label{fig:stereo_vision_bx_variation}
  \includegraphics[width=0.80\textwidth]{./img/fundamentacao/stereo_vision_bx_variation.png}
\end{figure}

Fonte: Autor

Figura \ref{fig:stereo_vision_bz_variation}

\begin{figure}[!htb]
  \centering
  \caption{Variação da \textit{baseline} ao longo do eixo z.}
  \label{fig:stereo_vision_bz_variation}
  \includegraphics[width=0.80\textwidth]{./img/fundamentacao/stereo_vision_bz_variation.png}
\end{figure}

Fonte: Autor

\pagebreak

\subsection{Resolução de profundidade em função da variação do ângulo entre câmeras.}

Figura \ref{fig:stereo_vision_arc_variation}

\begin{figure}[!htb]
  \centering
  \caption{Variação do ângulo entre câmeras.}
  \label{fig:stereo_vision_arc_variation}
  \includegraphics[width=0.80\textwidth]{./img/fundamentacao/stereo_vision_arc_variation.png}
\end{figure}
Fonte: Autor

\pagebreak

\subsection{Compensação da distância focal.}

A equação abaixo descreve a relação da resolução de profundidade $dZ_d$ com a distância total $Z$, distância focal $f$, o tamnho da\textit{baseline} $b$ e a precisão do sistema $dp_x$ \cite{Kyto2011}.

$$ dZ_d = \frac{Z^{2}}{fb}dp_x$$

Compensando a distância focal $f$ com o comprimento da \textit{baseline} $b$ é possível manter a resolução de profundidade.
É ilustrado na Figura \ref{fig:focal_dist_comp_baseline} como a ocupação dos sensores óticos $Z_L$ e $Z_R$ se mantém constante se houver uma compensação das distâncias focais $f_L$ e $f_R$ por meia da distância que separa as duas distâncias focais, $b_1$ e $b_2$.

\begin{figure}[!htb]
  \centering
  \caption{Compensação da diferença da distância focal por meio da variação da \textit{baseline}.}
  \label{fig:focal_dist_comp_baseline}
  \includegraphics[width=0.95\textwidth]{./img/fundamentacao/focal_dist_comp_baseline.png}
\end{figure}

Fonte: Autor

\pagebreak

Figura \ref{fig:aberturas} 

\begin{figure}[!htb]
  \centering
  \caption{FOV - 50mm vs Logitec.}
  \label{fig:aberturas}
  \includegraphics[width=0.80\textwidth]{./img/fundamentacao/aberturas.png}
\end{figure}

Fonte: Autor

\pagebreak

Figura \ref{fig:50m_and_logitec_equi} 

\begin{figure}[!htb]
  \centering
  \caption{Medidas da compensação da diferença da distância focal por meio da variação da \textit{baseline} utilizada no experimento.}
  \label{fig:50m_and_logitec_equi}
  \includegraphics[width=0.80\textwidth]{./img/fundamentacao/50m_and_logitec_equi.png}
\end{figure}

Fonte: Autor

\pagebreak

% \section{Sensores de imagem}

% \pagebreak

% \subsection{Canon 500D}

% \begin{itemize}
%     \item Medidas laterais do sensor: $22.3mm \times 14.9mm = 332.3m m^{2}$ 
    
%     \item Medida diagonal do sensor: $26.82mm$.

%     \item Fator de Corte: $1.61$.
    
%     \item Espaçamento entre pixel: $4.68\mu m$.
    
%     \item Área do pixel: $21.9\mu m^{2}$.
    
%     \item Densidade de pixel: $4.56 Mega Pixel/cm^{2}$. 
% \end{itemize}

% \pagebreak

% \ref{fig:focal_dist_test_1} 

% \begin{figure}[!htb]
%   \centering
%   \caption{Teste de distância focal.}
%   \label{fig:focal_dist_test_1}
%   \includegraphics[width=0.80\textwidth]{./img/focal_dist_test_1.png}
% \end{figure}
% Fonte: Autor

% \pagebreak


% \subsection{Logitec}

% \pagebreak

% \subsection{Calibração das Câmeras}

% \begin{figure}[!htb]
%   \centering
%   \caption{Representação Geométrica da Câmera.}
%   \label{fig:cammat_img}
%   \includegraphics[width=0.80\textwidth]{./img/cammat_img.png}
% \end{figure}
% Fonte: Autor

% \pagebreak

\section{Acionamento de motor DC}

\subsection{\textit{Tri Motor Shield}}

\cite{borges2019}

\begin{figure}[!htb]
  \centering
  \caption{Tri Motor Shield acoplado ao módulo Arduino Uno.}
  \label{fig:TriMotorShield}
  \includegraphics[width=0.60\textwidth]{./img/fundamentacao/TriMotorShield.jpg}
\end{figure}
Fonte: \textit{Borges \textit{Corporation}}

\pagebreak

\subsection{Ponte H L298N}

Como apresentado na Figura \ref{fig:L298N_block_diagram}, esse componente é composto de dois \textit{drivers} de corrente contínua de topologia Ponte H. Seu acionamento pode ser efetuado com padrão de nível lógico \textit{TTL}. A tensão máxima de operação é $50V$ e a corrente contínua máxima suportada é $2A$ \cite{stmicroelectronics2000}.

\begin{figure}[!htb]
  \centering
  \caption{Esquemático do L298N.}
  \label{fig:L298N_block_diagram}
  \includegraphics[width=0.95\textwidth]{./img/fundamentacao/L298N_block_diagram.png}
\end{figure}
Fonte: \textit{STMicroelectronics}


\chapter{Concepções}

\section{Concepção da plataforma de prototipagem}

Sistemas com \textit{linux} embarcado possibilitam o uso da biblioteca OpenCV \cite{Suryatali2015}. Além disso facilitam as práticas de prototipagem. Isso pois para atualizar um algorítimo específico, não há a necessidade de desligar o sistema para regravar o \textit{firmware}.
Porém processadores de baixo custo que rodam \textit{linux} embarcado, como os presentes nas placas \textit{Raspberry Pi}, possuem um tempo de resposta impreciso para acionar o GPIO. Logo um processador de tempo real é necessário para geração dos sinais PWM que irão controlar a potência fornecida aos motores. Por meio de uma comunicação serial processador rodando \textit{linux} embarcado envia o valor de potência para o processador de tempo real. Esse deve modular o valor de potência em valor de largura de pulso para acionamento do motores, como indicado na Figura \ref{fig:mapa_sys}. 

\begin{figure}[!htb]
  \centering
  \caption{Mapa do Sistema.}
  \label{fig:mapa_sys}
  \includegraphics[width=0.7\textwidth]{./img/concepcao/mapa_sys.png}
\end{figure}
Fonte: \textit{Autor}

\pagebreak

Com um acelerômetro de baixo custo se torna possível a implementação de uma interface de controle. 
Considerando que o acelerômetro seja capaz de medir valores constantes de aceleração nos três eixos, é possível usar a direção da gravidade da terra como referencial para medir a inclinação do mesmo. 
Logo é possível a implementação de um manche, pois com o acelerômetro acoplado no mesmo é possível identificar quando esse é inclinado. A Figura \ref{fig:controle_diferencial} ilustra a situação onde o valor da componente y da gravidade medidos pelo acelerômetro é associado ao valor da diferença de potência aplicada entre os motores. Assim se tem associado o raio da curva exercida á inclinação do acelerômetro.

\begin{figure}[!htb]
  \centering
  \caption{Controle de Curva.}
  \label{fig:controle_diferencial}
  \includegraphics[width=0.80\textwidth]{./img/concepcao/controle_diferencial.png}
\end{figure}
Fonte: \textit{Autor}

\pagebreak

A unidade de controle se comunica com uma interface de controle como apresentado na Figura \ref{fig:controle_para_atuador}.

\begin{figure}[!htb]
  \centering
  \caption{Interface de controle.}
  \label{fig:controle_para_atuador}
  \includegraphics[width=0.95\textwidth]{./img/concepcao/controle_para_atuador.png}
\end{figure}
Fonte: \textit{Autor}

Uma vez podendo controlar de forma independente a potência que é aplicada em cada motor, é possível exercer curvas de raios diferentes. 
Na Figura \ref{fig:angulo_dif_power} é exemplificado o caso onde a diferença da potência aplicada entre os motores do Carro A é menor que no Carro B.

\begin{figure}[!htb]
  \centering
  \caption{Diferença de potência aplicada aos motores.}
  \label{fig:angulo_dif_power}
  \includegraphics[width=0.7\textwidth]{./img/concepcao/angulo_dif_power.png}
\end{figure}
Fonte: \textit{Autor}

\pagebreak

A interface de controle pode ser implementada como apresentado na Figura \ref{fig:controle_acelerometro}. Um acelerômetro de três eixos é capaz de decompor o vetor de aceleração em duas componentes independentes. Dessa forma, além da componente y associada ao controle de curva, é possível associar a componente x ao controle de velocidade. Dessa forma, inclinando o manche na direção da componente x é possível controlar a potência que é aplicada em ambos os motores. Assim como inclinando o o manche na direção y é possível controlar a diferença de potência que é aplicada entre os motores. 

\begin{figure}[!htb]
  \centering
  \caption{Interface de Controle.}
  \label{fig:controle_acelerometro}
  \includegraphics[width=1\textwidth]{./img/concepcao/controle_acelerometro.png}
\end{figure}
Fonte: \textit{Autor}

Botões na interface de controle, como os botões A, B, C e D na Figura \ref{fig:controle_acelerometro}, possibilitam mais funcionalidades.
Para que a interface de controle não fique acionando os motores enquanto não estiver sendo manuseada, os botões podem ser utilizados para isso. Dessa forma, se for programado que a interface de controle só irá enviar sinais se os botões estiverem sendo pressionados, não acontecerá acionamentos em quanto a interface de controle não estiver sendo manuseada.

\pagebreak

A energia da plataforma provem de um barramento 12 V. Um conversor \textit{buck} pode ser usador para criar um barramento de 5 V. Dessa forma é possível fornecer energia a sistemas que trabalham com alimentação de 5 V, como as placas \textit{Raspberry Pi}. 
Uma ponte H controlada pelo processador de tempo real pode regular a potência fornecida aos motores pelo barramento de 12 V. O arranjo descrito é ilustrado na Figura \ref{fig:mapa_pot}.  

\begin{figure}[!htb]
  \centering
  \caption{Mapa de potência.}
  \label{fig:mapa_pot}
  \includegraphics[width=0.7\textwidth]{./img/concepcao/mapa_pot.png}
\end{figure}
Fonte: \textit{Autor}

\pagebreak

\section{Concepção do sensor de distância baseado em visão estereoscópico}

O acoplamento de um par de câmeras permite, além dos algorítimos de visão monocular, o teste de algorítimos de visão estereoscópica.
Nesse trabalho serão testados um sistema de visão estéreo montado com duas \textit{webcams} e o sistema Microsoft Kinect. 
Ambos os sistemas, por meio do mapa de profundidade composto pelas imagens do par de câmeras, terão função de medir a distância frontal como apresentado na Figura \ref{fig:visao_estereo}.

\begin{figure}[!htb]
  \centering
  \caption{Sensor de Distância.}
  \label{fig:visao_estereo}
  \includegraphics[width=0.5\textwidth]{./img/concepcao/visao_estereo.png}
\end{figure}
Fonte: \textit{Autor}

\pagebreak

\chapter{Projetos}

\section{Revisão da tecnologia disponível para o projeto}

Fundamentado no quesito disponibilidade do componente no DAELN (Departamento Acadêmico de Eletrônica), o \textit{hardware} escolhido tanto para o acionador dos motores quanto para o projeto da interface de controle foi o \textit{kit} de desenvolvimento \textit{Arduino Uno}. Esse carrega um microcontrolador \textit{AVR Atmega 328p} de \textit{8bits}.
Um \textit{Raspberry Pi 3 Model B+} havia disponível para ser usado como unidade principal de processamento, responsável por gerenciar os microcontroladores AVR e as câmeras. Esse também irá gerenciar a comunicação entre a rede \textit{wifi} e a plataforma além de fornecer a possibilidade de usar sua antena de comunicação \textit{Bluetooth}. Como fonte de energia foi utilizado um arranjo paralelo de bateria de chumbo-ácido de 12 V.
Um conversor estático do tipo \textit{buck} é responsável de gerar um barramento de tensão de 5 V a partir do barramento de 12 V onde as baterias estão conectadas. Haviam duas câmeras Logitec iguais que foram arranjadas em um sistema de visão estereoscópica e um sensor Microsoft Kinect. O Arranjo da plataforma é ilustrado na Figura \ref{fig:hardware_utilizado}. No caso da plataforma for esquipada com o sensor Kinect, esse deve ser alimentado com uma tensão de 12 V.

\begin{figure}[!htb]
  \centering
  \caption{\textit{Hardware} utilizado.}
  \label{fig:hardware_utilizado}
  \includegraphics[width=1\textwidth]{./img/projeto/hardware_utilizado.png}
\end{figure}
Fonte: \textit{Autor}

\pagebreak

\begin{figure}[!htb]
  \centering
  \caption{\textit{Hardware} utilizado.}
  \label{fig:hardware_utilizado_controle}
  \includegraphics[width=1\textwidth]{./img/projeto/hardware_utilizado_controle.png}
\end{figure}
Fonte: \textit{Autor}

\pagebreak

Os componentes eletro-eletrônico necessários:
\begin{enumerate}[label=\alph*)]
    \item 1 placa \textit{Raspberry Pi 3 Model B+} 
    \item 2 placas \textit{Arduino Uno}
    \item 1 placas \textit{Tri Motor Shield}
    \item 2 Motores DC.
    \item 1 placas \textit{Conversor de Tensão Buck Boost Ajustável DC-DC}
    \item 2 Webcams Logitec
    \item 1 Sensor Kinect
    \item 1 Placa GY-521
    \item 1 Notebook
\end{enumerate}

\pagebreak

\begin{figure}[!htb]
  \centering
  \caption{Diagrama Elétrico.}
  \label{fig:diagrama_eletrico}
  \includegraphics[width=1\textwidth]{./img/projeto/diagrama_eletrico.png}
\end{figure}
Fonte: \textit{Autor}
\pagebreak

\section{Projeto do controle}

\subsection{\textit{Hardware}}

Por meio da combinação de botões pressionados é possível acionar diferentes funcionalidades. 
A Figura \ref{fig:controle_acel} ilustra o acionamento da funcionalidade de controle do carro por meio do acelerômetro. O botão A pressionado faz com que o controle envie o sinal da componente y da gravidade medida pelo acelerômetro. 
O botão C pressionado faz com que a interface de controle envie o sinal da componente x. 
Logo um valor do tipo inteiro sem sinal de 8 bits é utilizado para representar a amplitude do sinal.
A polaridade do sinal de cada motor pode ser representada por um sinal booleano.
Como o valor máximo representado por uma variável de 8 bits é 255, a soma das amplitudes do sinal \textbf{power\_dif} e \textbf{power\_ref} não deve extrapolar esse valor. Logo quando ambos os botões A e C estiverem pressionados, o valor de amplitude do sinal das componentes da gravidade são divididos por dois. 

\begin{figure}[!htb]
  \centering
  \caption{Controle por meio do acelerômetro.}
  \label{fig:controle_acel}
  \includegraphics[width=1\textwidth]{./img/projeto/controle_acel.png}
\end{figure}
Fonte: \textit{Autor}

\pagebreak

Dessa forma , como indicado na Figura \ref{fig:controle_acel_individual}, é possível utilizar as componentes da aceleração da gravidade de forma independente. Como as componentes dessa vez não serão somadas, suas amplitude podem assumir valores maiores. No caso de utilizar apenas o botão A, o plataforma deve se mover em linha reta. 
Porém os motores não possuem mesmo rendimento.
Isso impossibilita um deslocamento em linha reta se for disponibilizada a mesma potência para ambos os motores. 
Uma margem de compensação \textbf{power\_comp}, no caso da Figura \ref{fig:controle_acel_individual}: $1/4$, da amplitude máxima do sinal pode ser deixada.
Dessa forma possibilitando implementado o sistema de controle afim de compensar a diferença de rendimento dos motores.

\begin{figure}[!htb]
  \centering
  \caption{controle acel individual.}
  \label{fig:controle_acel_individual}
  \includegraphics[width=1\textwidth]{./img/projeto/controle_acel_individual.png}
\end{figure}
Fonte: \textit{Autor}

\pagebreak

Como ilustrado na figura \ref{fig:controle_botao_up}.
Por meio dos botões B e D é possível acionar cada motor de forma independente afim de atribuir a máxima potência para cada motor. 
Dessa forma se torna possível controlar o carro apenas dosando a potência atribuída a cada motor ao pulsar os botões B e D.

\begin{figure}[!htb]
  \centering
  \caption{controle botao.}
  \label{fig:controle_botao_up}
  \includegraphics[width=0.75\textwidth]{./img/projeto/controle_botao_up.png}
\end{figure}
Fonte: \textit{Autor}

E a Figura \ref{fig:controle_botao_backward} ilustra que também á possibilidade de acionar os motores de forma inversa a Figura \ref{fig:controle_botao_up}.

\begin{figure}[!htb]
  \centering
  \caption{controle botao backward.}
  \label{fig:controle_botao_backward}
  \includegraphics[width=0.75\textwidth]{./img/projeto/controle_botao_backward.png}
\end{figure}
Fonte: \textit{Autor}

\pagebreak

\subsection{\textit{Software}}

O sistema ROS permite a fácil comunicação entre nós de processamento via TCP/IP.
Logo, uma vez utilizando o sistema ROS, o sistema de controle pode ser divido como no grafo na Figura \ref{fig:rosnet_controle}. 
O nó \textit{/steering\underline{\hspace{.0625in}}wheel} e o \textit{/motor\underline{\hspace{.0625in}}driver} rodam cada um em uma placa Arduino Uno e o nó de controle roda Raspberry Pi. O pacote ROSSERIAL possibilita aos Arduinos Unos o acesso a camada de transporte por meio da porta serial. Dessa forma esses conseguem se comunicar via protocolo TCP/IP.

\begin{figure}[!htb]
  \centering
  \caption{Rede de Controle.}
  \label{fig:rosnet_controle}
  \includegraphics[width=1\textwidth]{./img/projeto/rosnet_controle.png}
\end{figure}
Fonte: \textit{Autor}

Apêndice \ref{app:steering_wheel}

\pagebreak

\section{Projeto do sensor de distância baseado em visão estereoscópico}
    \subsection{\textit{Hardware}}
    \pagebreak
    \subsection{\textit{Firmware}}    
    \pagebreak
    
\section{Projeto do procedimento de Teste}
    \subsection{Elaboração da tarefa teste}
    \pagebreak
    
\chapter{Implementações}

\section{Implementação da plataforma de prototipagem}
    \subsection{\textit{Hardware}}
    \pagebreak
    \subsection{\textit{Firmware}}
    \pagebreak
    
\section{Implementação do sensor de distância baseado em visão estereoscópico}
    \subsection{\textit{Hardware}}
    \pagebreak
    \subsection{\textit{Firmware}}
    \pagebreak
    
\section{Implementação do procedimento de teste}

\chapter{Operação}

\section{Teste da plataforma de prototipagem}
    \subsection{Análise do consumo de energia}
    \pagebreak
    
\section{Teste do sensor de distância baseado em visão estereoscópico}
    \subsection{Verificação dos valores de distância medidos}
    \pagebreak
    
    \subsection{Verificação dos tempos de processamento}
    \pagebreak
    
\section{Execução do teste}
    \subsection{Verificação do desempenho mediante ao procedimento de teste}
    \pagebreak
    
\chapter{Conclusão}

\begin{figure}[!htb]
  \centering
  \caption{Composição do \textit{hardware} do sistema.}
  \label{fig:robotCar_hardware}
  \includegraphics[width=0.95\textwidth]{./img/robotCar_hardware.png}
\end{figure}
Fonte: Autor

\pagebreak

Na Figura \ref{fig:robotCar_software} é apresentado a composição de \textit{softwares} que compõem o sistema. O ROS possibilita que o processamento de imagem pode ser executado parcialmente ou completamento pelo \textit{hardware} de maior pode computacional (que no caso é o computador).

\begin{figure}[!htb]
  \centering
  \caption{Composição do \textit{software} do sistema.}
  \label{fig:robotCar_software}
  \includegraphics[width=0.95\textwidth]{./img/robotCar_software.png}
\end{figure}
Fonte: Autor

\pagebreak

\chapter{Implementação}

\chapter{Operação}
  \includegraphics[width=0.7\textwidth]{./img/angulo_dif_power.png}


 % \chapter{CONCLUSÃO}

% As conclusões devem responder às questões da pesquisa, em relação aos objetivos e hipóteses. Devem ser breves podendo apresentar recomendações e sugestões para trabalhos futuros.


% ----------------------------------------------------------
% ELEMENTOS PÓS-TEXTUAIS 
% ----------------------------------------------------------
\postextual

% ----------------------------------------------------------
% REFERÊNCIAS BIBLIOGRÁFICAS
% ----------------------------------------------------------
\bibliography{bib_eng_eletronic}

% ----------------------------------------------------------
% GLOSSARIO
% ----------------------------------------------------------
% Consulte o manual da classe abntex2 para orientações sobre o glossário.
%\glossary

% ----------------------------------------------------------
% APÊNDICES
% ----------------------------------------------------------
 %\begin{apendicesenv}
 \apendices
 \chapter{Código fonte da interface de controle}
% ----------------------------------------------------------
\label{app:steering_wheel}
	
\lstset{language=C++}
\begin{lstlisting}
#include<Wire.h>
#include <ros.h>
#include <std_msgs/Int8.h>
#include <std_msgs/UInt8.h>
#include <std_msgs/Bool.h>

#define USE_USBCON

//------------------------------------------------------------------------------
const int A = 8;//L1
const int B = 9;//L2
const int C = 10;//R1
const int D = 11;//R2

//Endereco I2C do MPU6050
const int MPU=0x68;

int8_t ACCEL_XOUT_H;
int8_t ACCEL_XOUT_L;
int8_t ACCEL_YOUT_H;
int8_t ACCEL_YOUT_L;
uint8_t B_CTRL;

ros::NodeHandle nh;	
std_msgs::Int8 power_ref, power_dif;
std_msgs::UInt8 ctrl_mode;

ros::Publisher pub_ctrl_mode("ctrl_mode", &ctrl_mode);
ros::Publisher pub_power_ref("ctrl_power_ref", &power_ref);
ros::Publisher pub_power_dif("ctrl_power_dif", &power_dif);

//------------------------------------------------------------------------------
void setup()
{
    pinMode(A, INPUT);
    pinMode(B, INPUT);
    pinMode(C, INPUT);
    pinMode(D, INPUT);

    nh.initNode();
    nh.advertise(pub_power_ref); 
    nh.advertise(pub_power_dif);
    nh.advertise(pub_ctrl_mode);

    Wire.begin();
    Wire.beginTransmission(MPU);
    Wire.write(0x6B);

    //Inicializa o MPU-6050
    Wire.write(0); 
    Wire.endTransmission(true); 
}


void loop()
{
    Wire.beginTransmission(MPU);
    Wire.write(0x3B);  // starting with register 0x3B (ACCEL_XOUT_H)
    Wire.endTransmission(false);
    
    //Solicita os dados do sensor
    Wire.requestFrom(MPU, 14, true); 

    //Armazena o valor dos sensores nas variaveis correspondentes
    ACCEL_XOUT_H = Wire.read(); //0x3B (ACCEL_XOUT_H)
    ACCEL_XOUT_L = Wire.read(); //0x3C (ACCEL_XOUT_L)     
    ACCEL_YOUT_H = Wire.read(); //0x3D (ACCEL_YOUT_H)
    ACCEL_YOUT_L = Wire.read(); //0x3E (ACCEL_YOUT_L)

    power_ref.data = ACCEL_XOUT_H;
    power_dif.data = ACCEL_YOUT_H;
    
    B_CTRL = 0x00;
    if(digitalRead(A)==LOW){
        B_CTRL = 0b00000001;
    }
    if(digitalRead(B)==LOW){
        B_CTRL |= 0b00000010;
    }
    if(digitalRead(C)==LOW){
        B_CTRL |= 0b00000100;
    }   
    if(digitalRead(D)==LOW){
        B_CTRL |= 0b00001000;
    }

    pub_power_ref.publish(&power_ref);
    pub_power_dif.publish(&power_dif);
    pub_ctrl_mode.publish(&ctrl_mode); 

    //---------------------------------------
    nh.spinOnce();
    delay(10); 	
}
\end{lstlisting}
\pagebreak 

\chapter{Código fonte do driver dos motores}
% ----------------------------------------------------------
\label{app:driver_node}
	
\lstset{language=C++}
\begin{lstlisting}
#include <avr/io.h>
#include <ros.h>
#include <std_msgs/Int8.h>
#include <std_msgs/UInt8.h>
#include <std_msgs/Bool.h>
#define USE_USBCON

ros::NodeHandle  nh;
	
//---------------------------------------
#define ENA1 PB0
#define IN11 PB1
#define IN12 PB2
#define ENB1 PB3
#define IN13 PB4
#define IN14 PB5

#define ENA2 PC0
#define IN21 PC1
#define IN22 PC2
#define ENB2 PC3
#define IN23 PC4
#define IN24 PC5

#define IN31 PD0
#define IN32 PD1
#define IN33 PD2
#define IN34 PD3
#define IN35 PD4
#define IN36 PD5
#define IN37 PD6
#define IN38 PD7

//---------------------------------------
std_msgs::UInt8 dutycycle_L_msg, dutycycle_R_msg;
uint8_t dutycycle_L, dutycycle_R;
std_msgs::Bool spin_L_msg, spin_R_msg;
bool spin_L, spin_R;

//---------------------------------------
void dutycycle_L_callback(const std_msgs::UInt8 &dutycycle_L_msg)
{
    dutycycle_L = dutycycle_L_msg.data;
}
ros::Subscriber<std_msgs::UInt8> sub_command_L("power_L", &dutycycle_L_callback);

//---------------------------------------
void spin_L_callback(const std_msgs::Bool &spin_L_msg)
{
    spin_L = spin_L_msg.data;

}
ros::Subscriber<std_msgs::Bool> sub_command_L("spin_L", &spin_L_callback);

//---------------------------------------
void dutycycle_R_callback(const std_msgs::UInt8 &dutycycle_R_msg)
{
    dutycycle_R = dutycycle_R_msg.data;
    if(dutycycle_R < 0){
      dutycycle_R = (-1)*dutycycle_R;
      spin_R = 0xff;
    }
    else{
      spin_R = 0x00;
    }
}
ros::Subscriber<std_msgs::UInt8> sub_command_R("power_R", &dutycycle_R_callback);

//---------------------------------------
void spin_R_callback(const std_msgs::Bool &spin_R_msg)
{
    spin_R = spin_R_msg.data;

}
ros::Subscriber<std_msgs::Bool> sub_command_L("spin_R", &spin_R_callback);

//---------------------------------------
void setup()
{
    dutycycle_L = 0x00;
    dutycycle_R = 0x00;

    DDRB = 0xFF;
    DDRC = 0xFF;
    DDRD = 0xFF;

    PCMSK2 = 0x00;
    UCSR0B = 0x00;

    PORTB = 0x00;
    PORTC = 0x00;
    PORTD = 0x00;
  
    nh.initNode();
    nh.subscribe(sub_command_L); 
    nh.subscribe(sub_command_R);
}

void loop()
{
    PORTB = 0x00;
    PORTC = 0x00;
    for(uint8_t i=0; i<256; i++)
    {     
      //---------------------------------------
      if(i > dutycycle_L)
        PORTC = 0x00;
      else
        if(spin_R)
          PORTC = 0b00101101;
        else
          PORTC = 0b00011011;

      //---------------------------------------
      if(i > dutycycle_R)
        PORTB = 0x00;      
      else
        if(spin_L)
          PORTB = 0b00101101;
        else
          PORTB = 0b00011011;

      //---------------------------------------
      nh.spinOnce(); 
    }   	
}
\end{lstlisting}
\pagebreak 

\chapter{Código fonte do sistema de controle}
% ----------------------------------------------------------
\label{app:controller}
	
\lstset{language=python}
\begin{lstlisting}
#!/usr/bin/env python
from __future__ import print_function
import rospy
from std_msgs.msg import Int8
from std_msgs.msg import UInt8
from std_msgs.msg import Bool
from geometry_msgs.msg import Point32

class Controller(object):
    """docstring for Controller"""
    def __init__(self):
        rospy.init_node('controller_py', anonymous=True)

        self.rate = rospy.Rate(40)#Hz

        self.ctrl_mode = 0
        self.power_ref = 0
        self.power_dif = 0
        self.power_ref_input = 0
        self.power_dif_input = 0 

        self.spin_L = True
        self.spin_R = True         
        self.power_L = 0
        self.power_R = 0

        self.stereo_vision_central_histogram = 0

    def signals_publisher_init(self):
        self.pub_power_L = rospy.Publisher('power_L', Int8, queue_size=10)
        self.pub_power_R = rospy.Publisher('power_R', Int8, queue_size=10)
        self.pub_spin_L = rospy.Publisher('spin_L', Bool, queue_size=10)
        self.pub_spin_R = rospy.Publisher('spin_R', Bool, queue_size=10)

    def signals_subscriber_init(self):
        rospy.Subscriber("ctrl_mode", UInt8, self.ctrl_mode_callback)
        rospy.Subscriber("ctrl_power_ref", Int8, self.power_ref_callback)
        rospy.Subscriber("ctrl_power_dif", Int8, self.power_dif_callback)  
        rospy.Subscriber("stereo_vision_central_depth_histogram", UInt8, self.stereo_vision_central_histogram_callback) 

    def ctrl_mode_callback(self, message):
        self.ctrl_mode = message.data

    def power_ref_callback(self, message):
        self.power_ref_input = message.data

    def power_dif_callback(self, message):
        self.power_dif_input = message.data

    def stereo_vision_central_histogram_callback(self, message):
        self.stereo_vision_central_histogram = message.data

    def main_loop(self):
        self.signals_subscriber_init()
        self.signals_publisher_init()
        
        while not rospy.is_shutdown():
            self.rate.sleep() 
            
            self.power_ref = 0 
            self.power_dif = 0               

            if self.ctrl_mode == 1:# A
                self.power_ref = 3*self.power_ref_input
                set_power = True
            elif self.ctrl_mode == 4:# C
                self.power_dif = 3*self.power_dif_input
                set_power = True           
            elif self.ctrl_mode == 5:# A and C
                self.power_ref = 3*self.power_ref_input
                self.power_dif = 3*self.power_dif_input
                set_power = True  
            elif self.ctrl_mode == 2:# B
                self.power_L = 192
                self.power_R = 0
                set_power = False 
            elif self.ctrl_mode == 8:# D
                self.power_L = 0
                self.power_R = 192
                set_power = False 
            elif self.ctrl_mode == 10:# B and D
                self.power_L = 192
                self.power_R = 192
                set_power = False
            elif self.ctrl_mode == 3:# A and B
                self.power_L = -192
                self.power_R = 0
                set_power = False 
            elif self.ctrl_mode == 12:# C and D
                self.power_L = 0
                self.power_R = -192
                set_power = False  
            elif self.ctrl_mode == 15:# A and B and C and D
                self.power_L = -192
                self.power_R = -192
                set_power = False 

            if set_power is True:
                self.driver_L = self.power_ref + self.power_dif + self.power_comp:
                if self.driver_L > 255:
                    self.driver_L = 255
                if self.driver_L < -255:
                    self.driver_L = -255

                if self.driver_L<0:
                    self.driver_L = np.uint8(-1*self.driver_L)
                    self.spin_L = False
                else:
                    self.driver_L = np.uint8(self.driver_L)
                    self.spin_L = True

                self.driver_R = self.power_ref - self.power_dif - self.power_comp:
                if self.driver_R > 255:
                    self.driver_R = 255
                if self.driver_R < -255:
                    self.driver_R = -255

                if self.driver_R<0:
                    self.driver_R = np.uint8(-1*self.driver_R)
                    self.spin_R = False
                else:
                    self.driver_R = np.uint8(self.driver_R)
                    self.spin_R = True

            print('stereo_vision_central_histogram:', self.stereo_vision_central_histogram)

            if self.stereo_vision_central_histogram > 200:
                print('break power!!!!')
                self.power_ref = np.uint8(0)
                self.power_dif = np.uint8(0)

            print(self.power_L, self.power_R)
            self.pub_spin_L.publish(self.spin_L)
            self.pub_spin_R.publish(self.spin_R)
            self.pub_power_L.publish(self.power_L)
            self.pub_power_R.publish(self.power_R)



# ======================================================================================================================
def controller():
    controller = Controller().main_loop()

if __name__ == '__main__':
    try:
        controller()
    except rospy.ROSInterruptException:
        pass

\end{lstlisting}
\pagebreak 
 %\end{apendicesenv}


% ----------------------------------------------------------
% ANEXOS
% ----------------------------------------------------------
%\begin{anexosenv}
\anexos
\chapter{Exemplificando um Anexo}
Texto do anexo aqui.

%\end{anexosenv}

%-----------------------------------------------------------
% INDICE REMISSIVO
%-----------------------------------------------------------
% \phantompart
% \printindex

\end{document}
